%%%%%%%%%%%%%%%%%%%%%%%%%%%%%%%%%%%%%%%%%
% University/School Laboratory Report
% LaTeX Template
% Version 3.1 (25/3/14)
%
% This template has been downloaded from:
% http://www.LaTeXTemplates.com
%
% Original author:
% Linux and Unix Users Group at Virginia Tech Wiki 
% (https://vtluug.org/wiki/Example_LaTeX_chem_lab_report)
%
% License:
% CC BY-NC-SA 3.0 (http://creativecommons.org/licenses/by-nc-sa/3.0/)
%
%%%%%%%%%%%%%%%%%%%%%%%%%%%%%%%%%%%%%%%%%

%----------------------------------------------------------------------------------------
%	PACKAGES AND DOCUMENT CONFIGURATIONS
%----------------------------------------------------------------------------------------

\documentclass{article}

\usepackage{graphicx} % Required for the inclusion of images
\usepackage{natbib} % Required to change bibliography style to APA
\usepackage{amsmath} % Required for some math elements 

\setlength\parindent{0pt} % Removes all indentation from paragraphs

\renewcommand{\labelenumi}{\alph{enumi}.} % Make numbering in the enumerate environment by letter rather than number (e.g. section 6)

%\usepackage{times} % Uncomment to use the Times New Roman font

%----------------------------------------------------------------------------------------
%	DOCUMENT INFORMATION
%----------------------------------------------------------------------------------------

\title{Amendments to the Dissertation\\``Analysis and Coordination of Mixed-criticality Cyber-physical Systems''} % Title

\author{Simon \textsc{Maurer}} % Author name

\date{\today} % Date for the report

\begin{document}

\maketitle % Insert the title, author and date

% If you wish to include an abstract, uncomment the lines below
% \begin{abstract}
% Abstract text
% \end{abstract}

%----------------------------------------------------------------------------------------
%	SECTION 1
%----------------------------------------------------------------------------------------

\section{Introduction}

This document provides written guidance of how and where I have implemented the required amendments.
This is done in Section~\ref{amendments}.
Further, in Section~\ref{changelog} I provide a complete list of implemented changes where also issues are covered that were briefly touched during the Viva but not mentioned in the list of amendments.

\section{Required and Suggested Amendments}
\label{amendments}
In this section I refer to the official list of amendments that were provided to me in written form.

\subsection{Clear Statement of the Thesis}
I changed the name of Section 1.1 from ``Research Questions'' to ``Thesis and Research Questions'' and added the thesis:
\emph{It is possible to bridge the gap between stream processing and Labelled Transition Systems (LTSs) for complex components.}

Further I introduced a new paragraph in the introduction hinting at the thesis (last paragraph on page 2).

Finally, I extended the section ``Contributions'' with the image I showed during the viva and a description of the image.

\subsection{Abstract}
In the abstract I now put the focus on bridging the gap between LTS and stream processing to highlight the novelty of the work.
Further, I highlight that the work spans from theory of a coordination model to the instantiation of the model as a coordination language.

\subsection{Claims to Novelty}
First, I extended the list in Section 1.2 of the dissertation to cover all major contributions and novelties of my work.
Second, when talking about PNSCs and SIAs I put more focus on SIAs, which is the novelty of the model and on the extensions described in Chapter 4.
Third, I explicitly point out the novelty of a concept when it is first introduced (e.g. semi-state message semantics).

\subsection{Typographical Slips}
I fixed spelling mistakes, consistency problems (hyphen, no hyphen), punctuation issues, subscripts of equations, and captions of figures.

\subsection{Lengthy Formal Definitions}
In order to make the formulas in Chapter 6 of the dissertation more accessible I added a brief informal description to each line.
Further I added a short description of the main features of Figures 6.8, 6.9, 6.10, and 6.14.

\subsection{Brief Description of Other Case Studies}
In Chapter 7 of the dissertation, I added a list of all simple examples that are available on GitHub and added a brief description for each of them.
I also provided a short documentation on how to compile and run the examples.

%----------------------------------------------------------------------------------------
%	SECTION 2
%----------------------------------------------------------------------------------------

\section{Changelog}
\label{changelog}
This sections provides an extensive changelog, listing the changes made to the dissertation after the examination of the 11th of January 2018.

\begin{itemize}
    \item All Chapters
        \begin{itemize}
            \item Fix typos, presentation, and figure captions throughout the dissertation
            \item Improve the readability by addressing smaller issues that surfaced during the examination
        \end{itemize}
    \item Abstract
        \begin{itemize}
            \item Change the focus to better address contributions and innovations
        \end{itemize}
    \item Chapter 1
        \begin{itemize}
            \item Put the focus on SIAs rather than PNSCs
        \end{itemize}
    \item Section 1.1
        \begin{itemize}
            \item Change the title to ``Thesis and Research Questions''
            \item Add a clear statement of the thesis
        \end{itemize}
    \item Section 1.2
        \begin{itemize}
            \item Add the figure from examination slides and a description of the figure
            \item Extend the list of contributions and made claim for new concepts
        \end{itemize}
    \item Subsection 1.2.1
        \begin{itemize}
            \item Chang the title to "Publications"
        \end{itemize}
    \item Section 2.2
        \begin{itemize}
            \item Hint at differences between Interface Automata and Synchronous Interface Automata
        \end{itemize}
    \item Section 2.3
        \begin{itemize}
            \item Move the deadlock and lonely blocker definitions from Chapter 5 to Section 2.3
            \item Add a crossroad example of a lonely blocker
        \end{itemize}
    \item Chapter 3
        \begin{itemize}
            \item Change the title to ``PNSC with SIA - An Analysable Event-based Component Model''
            \item Change the section structure and order to first group formal definitions and then provide examples
        \end{itemize}
    \item Section 3.1
        \begin{itemize}
            \item Fix the formula subscripts in Definition 3.1
            \item Change non-conflicting port condition to hold for n processes
            \item Change ``abstract process'' to ``composed process''
            \item Add an example of a composed process
        \end{itemize}
    \item Section 3.2
        \begin{itemize}
            \item Make the relation between a SIA and a PNSC process more explicit in definition 3.3
            \item Make clear that multiple state transitions can use the same action
            \item Mention that hidden actions do not require determinism
        \end{itemize}
    \item Subsection 3.2.3
        \begin{itemize}
            \item Make the relation between SIAs and PNSC processes more verbose
        \end{itemize}
    \item Section 4.1
        \begin{itemize}
            \item Improve wording to make the difference between different coupling mechanisms clear
        \end{itemize}
    \item Subsection 4.1.2
        \begin{itemize}
            \item Improve wording to make clear that decoupling inside a process is undesirable
            \item Fix the states of Figure 4.3 (need to be pairs)
            \item Fix the caption of Figure 4.3 (was the same as Fig 4.1)
            \item Add a forward reference to message types (loss and duplication of messages is tolerable in some cases)
            \item Remove the name ``smart'' FIFO
        \end{itemize}
    \item Subsection 4.2.2
        \begin{itemize}
            \item Improve the definition of the PBRT protocol
        \end{itemize}
    \item Subsection 4.4.1
        \begin{itemize}
            \item Highlight the process of higher criticality in Figure 4.13
        \end{itemize}
    \item Section 4.6
        \begin{itemize}
            \item Discuss the possible number of criticality levels
        \end{itemize}
    \item Chapter 5
        \begin{itemize}
            \item Improve the section structure throughout this chapter:
                \begin{itemize}
                    \item Discuss permanent blocking analysis on 2 processes then extend to n processes
                    \item Discuss deadlock analysis on 2 processes then extend to n processes
                    \item Describe specific aspects of the algorithms
                \end{itemize}
            \item Include more explicit talk about exponential growth of state space
        \end{itemize}
    \item Chapter 6
        \begin{itemize}
            \item Add informal descriptions of the lengthy formulas
        \end{itemize}
    \item Subsubsection 6.2.2.1
        \begin{itemize}
            \item Fix the FIFO length definition
        \end{itemize}
    \item Subsection 6.3.1
        \begin{itemize}
            \item Explain the choice of the names ``left'' and ``right'' for port collections
        \end{itemize}
    \item Chapter 7
        \begin{itemize}
            \item Add a list of examples that are provided on GitHub
            \item Make function names consistent with code on GitHub
            \item Change ``model checker'' to ``permanent blocking analysis''
        \end{itemize}
\end{itemize}

\end{document}
